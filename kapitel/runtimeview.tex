\section{Laufzeitsicht}
%Inhalt
%Diese Sicht beschreibt, wie sich die Bausteine des Systems als Laufzeitelemente (Prozesse, Tasks, Activities, Threads, ...)   verhalten und wie sie zusammenarbeiten.
%Als alternative Bezeichnungen finden Sie dafür auch:
%- Dynamische Sichten
%- Prozesssichten
%- Ablaufsichten
%Suchen Sie sich interessante Laufzeitszenarien heraus, z.B.:
%- Wie werden die wichtigsten Use-Cases durch die Architekturbausteine bearbeitet?
%- Welche Instanzen von Architekturbausteinen gibt es zur Laufzeit und wie werden diese gestartet, überwacht und beendet?
%- Wie arbeiten Systemkomponenten mit externen und vorhandenen Komponenten zusammen?
%- Wie startet das System (etwa: notwendige Startskripte, Abhängigkeiten von externen Subsystemen, Datenbanken, Kommunikationssystemen etc.)?
%
%Anmerkung: Kriterium für die Auswahl der möglichen Szenarien (d.h. Abläufe) des Systems ist deren Architekturrelevanz. Es geht nicht darum, möglichst viele Abläufe darzustellen, sondern eine angemessene Auswahl zu dokumentieren. 
%Kandidaten sind:
%1. Die wichtigsten 3-5 Anwendungsfälle
%2. Systemstart
%3. Das Verhalten an den wichtigsten externen Schnittstellen
%4. Das Verhalten in den wichtigsten Fehlerfällen
%
%Motivation
%Sie müssen (insbesondere bei objektorientierten Architekturen) nicht nur die Bausteine mit ihren Schnittstellen spezifizieren, sondern auch, wie Instanzen von Bausteinen zur Laufzeit miteinander kommunizieren.
%Form
%Dokumentieren Sie die ausgesuchten Laufzeitszenarien mit UML-Sequenz-, Aktivitäts-, oder Kommunikationsdiagrammen.
%Mit Objektdiagrammen können Sie Schnappschüsse der existierenden Laufzeitobjekte darstellen und auch instanziierte Beziehungen. Die UML bietet dabei die Möglichkeit zwischen aktiven und passiven Objekten zu unterscheiden. 
\subsection{Laufzeitszenario 1}
%- Laufzeitdiagramm
%- Erläuterung der Besonderheiten bei dem Zusammenspiel der Bausteininstanzen in diesem Diagramm

\subsection{Laufzeitszenario n}
