\section{Bausteinsicht}
%Inhalt
%Statische Zerlegung des Systems in Bausteine (Module, Komponenten, Subsysteme, Teilsysteme, Klassen, Interfaces, Pakete, Bibliotheken, Frameworks, Schichten, Partitionen, Tiers, Funktionen, Makros, Operationen, Datenstrukturen...) sowie deren Beziehungen.
%Motivation
%Dies ist die wichtigste Sicht, die in jeder Architekturdokumentation vorhanden sein muss. Wenn Sie es mit dem Hausbau vergleichen ist das der Grundrissplan.
%Form
%Die Bausteinsicht ist eine hierarchische Sammlung von BlackBox- und White-Box- Beschreibungen (siehe Abbildung unten):
%
%Ebene 1 ist die White-Box-Beschreibung des Gesamtsystems (System under Development / SUD) mit den Black- Box- Beschreibungen der Bausteine des Gesamtsystems
%Ebene 2 zoomt dann in die Bausteine der Ebene 1 hinein und ist somit die Sammlung aller White-Box- Beschreibungen der Bausteine der Ebene 1 zusammen mit den Black-Box-Beschreibungen der Bausteine der Ebene 2.
%Ebene 3 zoomt in die alle Bausteine der Ebene 2 hinein, u.s.w.
%Die Gliederung dieses Kapitels sieht dann folgendermaßen aus:
%============================
%White-Box-Template:
%Enthält mehrere Bausteine, zu denen Sie jeweils eine Black-Box Beschreibung machen.
%Ein- oder mehrere Black-Box-Templates:
%Für jeden Baustein aus dem White-Box-Template sollten folgende Angaben gemacht werden: (Kopieren Sie diese Punkte in die folgenden Unterkapitel)
%- Zweck / Verantwortlichkeit:
%- Schnittstelle(n):
%- Erfüllte Anforderungen:
%- Ablageort / Datei:
%- Sonstige Verwaltungsinformation: Autor, Version, Datum, Änderungshistorie

\subsection{Ebene 1}
%An dieser Stelle beschreiben Sie die White-Box-Sicht der Ebene 1 gemäß dem Whitebox-Template. Die Struktur ist im folgenden bereits vorgegeben.
%Das Überblicksbild zeigt das Innenleben des Gesamtsystems in Form der Bausteine 1 - n, sowie deren Zusammenhänge und Abhängigkeiten.
%Sinnvoll sind hier auch Beschreibungen der wichtige Begründungen, die zu dieser Struktur führen, insbesondere die Beschreibung der Abhängigkeiten (Beziehungen) zwischen den Bausteinen dieser Ebene.
%Evtl. verweisen Sie auch auf verworfene Alternativen (mit der Begründung, warum es verworfen wurde
%Die folgende Abbildung zeigt die Hauptbausteine unseres Systems und deren Abhängigkeiten.
%<hier Überblicksdiagramm einfügen>



\subsubsection{Bausteinname 1.1 (BlackBox Beschreibung)}
%Struktur gemäß Black-Box- Template:
%1* Zweck / Verantwortlichkeit:
%2* Schnittstelle(n):
%3* Erfüllte Anforderungen:
%4* Variabilität:
%5* Leistungsmerkmale:
%6* Ablageort / Datei:
%7* Sonstige Verwaltungsinformation:
%8* Offene Punkte:

%< Hier Überblicksdiagramm für Innenleben von Baustein 1 einfügen>

\subsubsection{ Bausteinname 2 (Whitebox-Beschreibung)}

%1 Überblicksdiagramm 

\subsection{Ebene 2}
%- zeigt das Innenleben des Bausteines in Diagrammform mit den lokalen Bausteinen 1 - n, sowie deren Zusammenhänge und Abhängigkeiten.
%- beschreibt wichtige Begründungen, die zu dieser Struktur führen
%- verweist evtl. auf verworfene Alternativen (mit der Begründung, warum es verworfen wurde

\subsubsection{Bausteinname 2.1 (BlackBox Beschreibung)}

\subsubsection{Bausteinname 3 (Whitebox-Beschreibung)}

%< Hier Überblicksdiagramm einfügen>

\subsection{Ebene 3}
%An dieser Stelle beschreiben Sie die White-Box- Sichten aller Bausteine der Ebene 2 als Folge von White-Box-Templates. Die Struktur ist identisch mit der Struktur auf Ebene 2. Kopieren Sie die entsprechenden Gliederungspunkte hierhier.
%Bei tieferen Gliederungen der Architektur kopieren Sie bitte das ganze Kapitel für die nächsten Ebenen.

\subsection{Beschreibung der Beziehungen}

\subsection{Offene Punkte}
