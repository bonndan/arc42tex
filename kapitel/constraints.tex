\section{Randbedingungen}
%(engl.: Architecture Constraints)
%Inhalt
%Fesseln, die Software-Architekten in ihren Freiheiten bezüglich des Entwurfs oder des Entwicklungsprozesses einschränken.
%Motivation
%Architekten sollten klar wissen, wo Ihre Freiheitsgrade bezüglich Entwurfsentscheidungen liegen und wo sie Randbedingungen beachten müssen.
%Randbedingungen können vielleicht noch verhandelt werden, zunächst sind sie aber da.
%Form
%Informelle Listen, gegliedert nach den Unterpunkten dieses Kapitels.
%Beispiele
%siehe Unterkapitel
%Hintergründe
%Im Idealfall sind Randbedingungen durch die Anforderungen vorgegeben, spätestens die Architekten müssen sich dieser Randbedingungen bewusst sein. 
%Den Einfluss von Randbedingungen auf Software- und Systemarchitekturen beschreibt  [Hofmeister+1999] (Softwware-Architecture, A Practical Guide, Addison-Wesley 1999) unter dem Stichwort „Global Analysis“.

\subsection{Technische Randbedingungen}
%Inhalt
%Tragen Sie hier alle technischen Randbedingungen ein. Zu dieser Kategorie gehören Hard- und Software-Infrastruktur, eingesetzte Technologien (Betriebssysteme, Middleware, Datenbanken, Programmiersprachen, ...).

\subsubsection{Hardware-Vorgaben}

%<hier Randbedingungen einfügen> 

\subsubsection{Software-Vorgaben}

%<hier Radbedingungen einfügen>

\subsubsection{Vorgaben des Systembetriebs}

%<hier Randbedingungen einfügen>
\subsubsection{Programmiervorgaben}

%<hier Randbedingungen einfügen>
%
%Beispiele
%
%Randbedingung
%Erläuterung
%Hardware-Infrastruktur
%Prozessoren, Speicher, Netzwerke, Firewalls und andere relevante Elemente der Hardware- Infrastruktur
%Software-Infrastruktur
%Betriebssysteme, Datenbanksysteme, Middleware, Kommunikationssysteme, Transaktionsmonitor, Webserver, Verzeichnisdienste
%Systembetrieb
%Batch- oder Onlinebetrieb des Systems oder notwendiger externer Systeme?
%Verfügbarkeit der Laufzeitumgebung
%Rechenzentrum mit 7x24h Betriebszeit?
%Gibt es Wartungs- oder Backupzeiten mit eingeschränkter Verfügbarkeit des Systems oder wichtiger Systemteile?
%Grafische Oberfläche
%Existieren Vorgaben hinsichtlich grafischer Oberfläche (Style Guide)?
%Bibliotheken, Frameworks und Komponenten
%Sollen bestimmte „Software-Fertigteile“ eingesetzt werden?
%Programmiersprachen
%Objektorientierte, strukturierte, deklarative oder
%Regelsprachen, kompilierte oder interpretierte
%Sprachen?
%Referenzarchitekturen
%Gibt es in der Organisation vergleichbare oder übertragbare Referenzprojekte?
%Analyse- und Entwurfsmethoden
%Objektorientierte oder strukturierte Methoden?
%Datenstrukturen
%Vorgaben für bestimmte Datenstrukturen, Schnittstellen zu bestehenden Datenbanken oder Dateien
%Programmierschnittstellen
%Schnittstellen zu bestehenden Programmen
%Programmiervorgaben
%Programmierkonventionen, fester Programmaufbau
%Technische Kommunikation
%Synchron oder asynchron, Protokolle
%Betriebssystem und Middleware
%Vorgegebene Betriebssysteme oder Middleware

\subsection{Organisatorische Randbedingungen}
%Inhalt
%Tragen Sie hier alle organisatorischen, strukturellen und ressourcenbezogenen Randbedingungen ein. Zu dieser Kategorie gehören auch Standards, die Sie einhalten müssen und juristische Randbedingungen.

\subsubsection{Organisation und Struktur}
%Organisation und Struktur
%Organisationsstruktur
%beim Auftraggeber
%Droht Änderung von Verantwortlichkeiten? 
%Änderung von Ansprechpartnern 
%Organisationsstruktur
%des Projektteams
%mit/ohne Unterauftragnehmer 
%Entscheidungsbefugnis der Projektleiterin 
%Entscheidungsträger
%Erfahrung mit ähnlichen Projekten 
%Risiko-/Innovationsfreude 
%Bestehende Partnerschaften oder
%Kooperationen
%Hat die Organisation bestehende Kooperationen mit bestimmten Softwareherstellern? 
%Solche Partnerschaften geben oftmals Produktentscheidungen (unabhängig von Systemanforderungen) vor. 
%Eigenentwicklung
%oder externe Vergabe
%Selbst entwickeln oder an externe Dienstleister vergeben?
%Entwicklung als Produkt oder zur eigenen
%Nutzung?
%Bedingt andere Prozesse bei Anforderungsanalyse und Entscheidungen. Im Fall der Produktentwicklung: 
%Neues Produkt für neuen Markt? 
%Verbessertes Produkt für bestehenden Markt? 
%Vermarktung eines bestehenden (eigenen) Systems 
%Entwicklung ausschließlich zur eigenen Nutzung?

%<hier Randbedingungen einfügen> 

\subsubsection{Ressourcen (Budget, Zeit, Personal)}
%Festpreis oder Zeit/Aufwand?
%Festpreisprojekt oder Abrechnung nach Zeit und
%Aufwand?
%Zeitplan
%Wie flexibel ist der Zeitplan? Gibt es einen festen Endtermin? Welche Stakeholder bestimmen den Endtermin?
%Zeitplan und Funktionsumfang
%Was ist höher priorisiert, der Termin oder der Funktionsumfang?
%Release-Plan
%Zu welchen Zeitpunkten soll welcher Funktionsumfang in Releases/Versionen zur Verfügung stehen?
%Projektbudget
%Fest oder variabel? In welcher Höhe verfügbar?
%Budget für technische
%Ressourcen
%Kauf oder Miete von Entwicklungswerkzeugen
%(Hardware und Software)?
%Team
%Anzahl der Mitarbeiter und deren Qualifikation,
%Motivation und kontinuierliche Verfügbarkeit.
%Team
%Anzahl der Mitarbeiter und deren Qualifikation,
%Motivation und kontinuierliche Verfügbarkeit.


\subsubsection{Organisatorische Standards}
%Vorgehensmodell
%Vorgaben bezüglich Vorgehensmodell? Hierzu gehören auch interne Standards zu Modellierung, Dokumentation und Implementierung.
%Qualitätsstandards
%Fällt die Organisation oder das System in den Geltungsbereich von Qualitätsnormen (wie ISO-9000)?
%Entwicklungs­ werkzeuge
%Vorgaben bezüglich der Entwicklungswerkzeuge
%(etwa: CASE-Tool, Datenbank, Integrierte Entwicklungsumgebung, Kommunikationssoftware, Middleware, Transaktionsmonitor).
%Konfigurations- und
%Versionsverwaltung
%Vorgaben bezüglich Prozessen und Werkzeugen
%Testwerkzeuge und prozesse
%Vorgaben bezüglich Prozessen und Werkzeugen
%Abnahme- und
%Freigabeprozesse
%Datenmodellierung und Datenbankdesign 
%Benutzeroberflächen 
%Geschäftsprozesse (Workflow) 
%Nutzung externer Systeme (etwa: schreibender Zugriff bei externen Datenbanken) 
%Service Level
%Agreements
%Gibt es Vorgaben oder Standards hinsichtlich Verfügbarkeiten oder einzuhaltender Service-Levels?


\subsubsection{Juristische Faktoren}
%Haftungsfragen
%Hat die Nutzung oder der Betrieb des Systems mögliche rechtliche Konsequenzen? 
%Kann das System Auswirkung auf Menschenleben oder Gesundheit besitzen? 
%Kann das System Auswirkungen auf Funktionsfähigkeit externer Systeme oder Unternehmen besitzen? 
%Datenschutz
%Speichert oder bearbeitet das System „schutzwürdige“ Daten?
%Nachweispflichten
%Bestehen für bestimmte Systemaspekte juristische Nachweispflichten?
%Internationale Rechtsfragen
%Wird das System international eingesetzt? 
%Gelten in anderen Ländern eventuell andere juristische Rahmenbedingungen für den Einsatz (Beispiel: Nutzung von Verschlüsselungsverfahren)? 
  
\subsection{Konventionen}
%Inhalt
%Fassen Sie unter dieser Überschrift alle Konventionen zusammen, die für die Entwicklung der Software-Architektur relevant sind.
%Form
%Entweder die Konventionen als Kapitel hier direkt einhängen oder aber auf entsprechende Dokumente verweisen.
%Beispiele
%- Programmierrichtlinien
%- Dokumentationsrichtlinien
%- Richtlinien für Versions- und Konfigurationsmanagement
%- Namenskonventionen